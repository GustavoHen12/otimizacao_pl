\documentclass{article}
\usepackage{amsfonts}
\usepackage{amsmath}
\usepackage{hyperref}
\usepackage[utf8]{inputenc}
\usepackage[portuguese]{babel}

\title{Trabalho Prático de Otimização - Parte 1}
\author{
    Gustavo Henrique da Silva Barbosa
    \and
    Pietro Polinari Cavassin
}

\begin{document}
\maketitle

\section{O Problema}
\begin{itemize}
\item Uma empresa de transporte recebeu uma carga a ser transportada.
A carga tem $n$ itens, $I_1, I_2, \dots, I_n$, cada um com seu peso 
$w_i$, para $1 \le i \le n$, em kg

\item A empresa tem apenas um caminhão com capacidade máxima de carga $C$ kg,
que fará uma ou mais viagens.

\item As viagens recebem números de $1$ a $k$, onde $k$ é o número de viagens
feitas.

\item Determinamos a função $v$, onde $v(i)$ é a viagem onde foi transportado
o item $I_i$. 

\item Para cada par ordenado $(a, b)$ entregue pelo cliente é determinada a
restrição $v(a) < v(b)$.

\item A tarefa é decidir quais itens vão em que viagem de forma a minimizar
o número de viagens.
\end{itemize}

\newpage
\section{Modelagem}

$$\min \sum_{j=1}^n y_j$$
S.A 

\begin{align}
\sum_{i=1}^n M_{ij} \cdot w_i \le C \cdot y_j&, & 
    \textsf{ para todo }& j \in \mathbb{N},\ 1 \le j \le n\\
\sum_{j=1}^n M_{ij} = 1&,             & 
    \textsf{ para todo }& i \in \mathbb{N},\ 1 \le i \le n
\end{align}
Para cada par ordenado $(a,b)$:
\begin{align}
\sum_{j=1}^p M_{aj} \ge \sum_{j=1}^p M_{bj}&, & 
    \textsf{ para todo }& p \in \mathbb{N},\ 1 \le p \le n\\
M_{aj} + M_{bj} \le 1&, &
    \textsf{ para todo }& j \in \mathbb{N},\ 1 \le j \le n
\end{align}
Restrições de valor:
\begin{align}
0 \le M_{ij}\le 1&, &
    \textsf{ para todo }& 1 \le i,j \le n\\
0 \le y_{j}\le 1&, &
    \textsf{ para todo }& 1 \le j \le n
\end{align}

As seguintes variáveis foram utilizadas na modelagem:
\begin{itemize}
    \item $C$: carga suportada pelo caminhão
    
    \item $y_j$: vale $0$ se a viagem $j$ não for feita e $1$ caso contrário.

    \item $M_{ij}$: vale $1$ se o item $i$ for transportado na viagem $j$
    e $0$, caso contrário.

    \item $w_i$: constante que equivale ao peso do item $i$.
    
    \item $n$: número de itens a serem transportados pelo caminhão
\end{itemize}

A restrição $(1)$ garante que, para cada viagem $j$, a soma das cargas a serem transportadas nela será menor ou igual à carga máxima $C$. A multiplicação dentro do somatório ($M_{ij} \cdot w_i$) garante que apenas sejam somados os pesos que serão transportados na viagem $j$. Da mesma forma, a multiplicação ao lado direito ($C \cdot y_j$) reduz a carga suportada a $0$, caso a viagem $j$ não seja feita ($y_j = 0$).

A restrição $(2)$ garante que cada item $i$ seja transportado em apenas uma viagem, visto que uma linha $i$ corresponde a um item.

A restrição $(3)$ garante que, para um par ordenado $(a,b)$, $v(a) \le v(b)$. Isso ocorre, pois para uma dada linha $a$, A soma dos itens de a entre a coluna $1$ e uma coluna $p$ será $1$ caso $v(a) <= p$ e $0$, caso contrário. Portanto, a restrição $(3)$ indica que $v(a) \le v(b)$

A restrição $(4)$ garante que, para um par ordenado $(a,b)$, o item $a$ seja transportado na mesma viagem que $b$. Isso ocorre, pois as colunas representam as viagens, e a soma das linhas $a$ e $b$ em uma mesma coluna não pode passar de 1. Ou seja, ambos os valores não podem ser $1$ ao mesmo tempo.

As restrições $(5)$ e $(6)$ limitam os valores dos itens de $M$ e $y$
entre $0$ e $1$.

\begin{section}{Instruções de compilação e execução}
    
        Para executar o programa, pode ser necessário utilizar a variável de ambiente LD\_LIBRARY\_PATH para especificar o diretório da biblioteca do lp\_solve. Dessa forma, antes de rodar o programa, copiar a seguinte linha no terminal (dentro do diretório do trabalho): 

    \texttt{\$ export LD\_LIBRARY\_PATH = .}

    Depois, compilar com 
    
    \texttt{\$ make}

    e executar com 
    
    \texttt{\$ ./parcial\_relaxada < teste\_1.in}
\end{section}

\begin{section}{Referências}

Bin Packing Problem. In: Wikipedia: the free encyclopedia. Disponível em: \url{https://en.wikipedia.org/wiki/Bin_packing_problem} Acesso em: 12 mar 2022. 

\end{section}

\end{document}